\documentclass[
  %fleqn,     das ist für die zentrierung
	parskip=half,
	captions=tableheading,
  titlepage=firstiscover, 		%************************************************** by Rk
	bibliography=totoc		%*************************************by Rk
	]{scrartcl}			
% \usepackage{etex}
% \reserveinserts{28}

%Das ist für die Kopfzeile
\usepackage[headsepline]{scrlayer-scrpage}
\pagestyle{scrheadings}
\clearpairofpagestyles
\ofoot{\pagemark}
\ohead{\headmark}
\automark{section}

% Warnung, falls nochmal kompiliert werden muss		%*************************************by Rk
\usepackage[aux]{rerunfilecheck}

% unverzichtbare Mathe-Befehle
\usepackage{amsmath}
% viele Mathe-Symbole
\usepackage{amssymb}
% Erweiterungen für amsmath
\usepackage{mathtools}
\usepackage{upgreek}
% Fonteinstellungen
\usepackage{fontspec}				%************************************************** by Rk
% Latin Modern Fonts werden automatisch geladen
% Latin Modern Fonts werden automatisch geladen
% Alternativ zum Beispiel:
%\setromanfont{Libertinus Serif}
%\setsansfont{Libertinus Sans}
%\setmonofont{Libertinus Mono}

\usepackage{polyglossia}
\usepackage[				%************************************************** by Rk
	backend=biber,
]{biblatex}  	
%Quellendatenbank	
\setmainlanguage{german}			
\addbibresource{lit.bib}		%************************************************** by Rk


\usepackage{expl3}
\usepackage{xparse}

\usepackage{physics}

\usepackage[unicode, german]{hyperref}
\usepackage[autostyle]{csquotes}
\usepackage[
  math-style=ISO,    % ┐
  bold-style=ISO,    % │
  sans-style=italic, % │ ISO-Standard folgen
  nabla=upright,     % │
  partial=upright,   % ┘
  warnings-off={           % ┐
    mathtools-colon,       % │ unnötige Warnungen ausschalten
    mathtools-overbracket, % │
  },                       % ┘
]{unicode-math}

% traditionelle Fonts für Mathematik
\setmathfont{Latin Modern Math}
% Alternativ zum Beispiel:
%\setmathfont{Libertinus Math}

\setmathfont{XITS Math}[range={scr, bfscr}]
\setmathfont{XITS Math}[range={cal, bfcal}, StylisticSet=1]

% Zahlen und Einheiten
\usepackage[
  locale=DE,                 % deutsche Einstellungen
  separate-uncertainty=true, % immer Fehler mit \pm
  per-mode=symbol-or-fraction,       % ^-1 für inverse Einheiten
  % output-decimal-marker=.,   % . statt , für Dezimalzahlen
]{siunitx}

% chemische Formeln
\usepackage[
  version=4,
  math-greek=default, % ┐ mit unicode-math zusammenarbeiten
  text-greek=default, % ┘
]{mhchem}

% Wenn man andere Schriftarten gesetzt hat,
% sollte man das Seiten-Layout neu berechnen lassen
\recalctypearea{}				%************************************************** by Rk


% richtige Anführungszeichen 
\usepackage[autostyle]{csquotes}

% schöne Brüche im Text
\usepackage{xfrac}

% Grafiken können eingebunden werden
\usepackage{graphicx}
% größere Variation von Dateinamen möglich
%\usepackage{grffile}
\usepackage{scrhack}

% Verbesserungen am Schriftbild
\usepackage{microtype}

% Standardplatzierung für Floats einstellen
\usepackage{float}
\usepackage[section, below]{placeins}
% \usepackage[..]{caption}
\floatplacement{figure}{htbp}
\floatplacement{table}{htbp}

\usepackage{booktabs}
\usepackage{subcaption}
\author{%
  Raphael Rico Kaiser\\%
  \href{raphael.kaiser@tu-dortmund.de}{raphael.kaiser@tu-dortmund.de}%
  \texorpdfstring{\and}{,}%
  Hendrik Trojan\\%
  \href{hendrik.trojan@tu-dortmund.de}{hendrik.trojan@tu-dortmund.de}%
}

\publishers{TU Dortmund - Fakultät Physik}
\usepackage{romannum}
\AtBeginDocument{\pagenumbering{arabic}}

\NewDocumentCommand \e {}
{
  \symup{e}
}

\NewDocumentCommand \const {}
{
  \text{const.}
}

\NewDocumentCommand \fig {mmm}
{
\begin{figure}
    \centering 
    \includegraphics[width=9cm]{#1}
    \caption{#3}
    \label{#2}
   \end{figure}
\nocite{*}
}
